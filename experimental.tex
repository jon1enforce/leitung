\documentclass[10pt,a4paper]{article}
\usepackage[utf8]{inputenc}
\usepackage[german]{babel}
\usepackage{geometry}
\usepackage{listings}
\usepackage{xcolor}
\usepackage{hyperref}

\geometry{a4paper, margin=2.5cm}

\definecolor{codegreen}{rgb}{0,0.6,0}
\definecolor{codegray}{rgb}{0.5,0.5,0.5}
\definecolor{codepurple}{rgb}{0.58,0,0.82}
\definecolor{backcolour}{rgb}{0.95,0.95,0.92}

\lstdefinestyle{mystyle}{
    backgroundcolor=\color{backcolour},   
    commentstyle=\color{codegreen},
    keywordstyle=\color{magenta},
    numberstyle=\tiny\color{codegray},
    stringstyle=\color{codepurple},
    basicstyle=\ttfamily\footnotesize,
    breakatwhitespace=false,         
    breaklines=true,                 
    captionpos=b,                    
    keepspaces=true,                 
    numbers=left,                    
    numbersep=5pt,                  
    showspaces=false,                
    showstringspaces=false,
    showtabs=false,                  
    tabsize=2
}

\lstset{style=mystyle}

\title{Sichere VoIP-Kommunikation - Protokollspezifikation}
\author{Entwicklerversion 2.0}
\date{\today}

\begin{document}

\maketitle

\begin{abstract}
Dieses Dokument beschreibt das erweiterte Protokoll für sichere VoIP-Kommunikation mit Fokus auf experimentelle Sicherheitsfeatures, RNG-Verifikation und Angriffserkennung. Das Protokoll lässt bewusst bestimmte Angriffsvektoren offen, um Angreiferverhalten zu analysieren.
\end{abstract}

\tableofcontents

\section{Protokoll-Übersicht}

\subsection{Kernprinzipien}
\begin{itemize}
\item \textbf{SIP-basiert} mit erweiterten Sicherheitsfeatures
\item \textbf{Einheitliches Framing} für alle Nachrichten (4-Byte Header + Body)
\item \textbf{Verify-Code System} zum Schutz gegen Replay-Angriffe
\item \textbf{Quantenresistente Hashes} (SHA3-256) für Merkle Trees
\item \textbf{Hybride Verschlüsselung} (RSA-4096 + AES-256-CBC)
\end{itemize}

\section{RNG-Verifikationssystem}

\subsection{Zweck und Design}
Das Verify-Code-System schützt SIP-Nachrichten, die von Natur aus unsicher sind. In der Experimentierphase wird der Seed (Client-Name) bewusst nicht geheim gehalten, um Angriffe auf den Initialisierungsvektor zu ermöglichen.

\subsection{Technische Implementierung}

\begin{lstlisting}[language=Python,caption=Verify-Code-Generator]
class VerifyGenerator:
    def __init__(self, seed, client_id=None):
        self.seed = str(seed)  # Öffentlicher Client-Name
        self.client_id = client_id
        self.counter = 0
        self._lock = threading.Lock()
    
    def generate_verify_code(self):
        base_string = f"{self.seed}:{self.counter}"
        hash_obj = hashlib.sha256(base_string.encode())
        code = hash_obj.hexdigest()[:4]
        self.counter += 1
        return code
\end{lstlisting}

\subsection{Angriffserkennung}

\begin{lstlisting}[language=Python,caption=Erweiterte Verify-Code-Validierung]
def verify_code_with_detection(received_code, client_info):
    # Normale Validierung zuerst
    if generator.verify_code(received_code):
        return "valid"
    
    # Prüfe auf Wiederverwendung (RNG-State-Angriff)
    for offset in range(1, 100):
        test_counter = current_counter - offset
        expected_code = calculate_expected_code(test_counter)
        if received_code == expected_code:
            log_rng_attack(client_info, received_code)
            return "reused"
    
    return "invalid"
\end{lstlisting}

\subsection{Protokollierung von Angriffen}

\begin{lstlisting}[language=Python,caption=Angriffsprotokollierung]
def log_rng_attack(client_info, code):
    log_entry = (
        "🚨 [RNG_STATE_ATTACK] VERIFY_CODE_REUSE\n"
        f"   Client: {client_info}\n"
        f"   Code: {code}\n"
        f"   ANALYSIS: Angreifer verwendet bekannten RNG State!\n"
        f"   ANALYSIS: Code wurde doppelt verwendet!\n"
    )
    with open("attacks.log", "a") as f:
        f.write(log_entry)
\end{lstlisting}

\section{Sicherheitsfeatures}

\subsection{Einheitliches Framing}

\begin{lstlisting}[language=Python,caption=Frame-Format]
def send_frame(sock, data):
    # 4-Byte Header (Network Byte Order)
    header = struct.pack('!I', len(data))
    full_message = header + data
    return sock.sendall(full_message)
\end{lstlisting}

\subsection{Quantensichere Merkle Trees}

\begin{lstlisting}[language=Python,caption=Merkle-Tree-Berechnung]
def build_merkle_tree(data_blocks):
    tree = [quantum_safe_hash(block) for block in data_blocks]
    while len(tree) > 1:
        if len(tree) % 2 != 0:
            tree.append(tree[-1])
        tree = [quantum_safe_hash(tree[i] + tree[i+1]) 
               for i in range(0, len(tree), 2)]
    return tree[0]
\end{lstlisting}

\subsection{Hybride Verschlüsselung}

\begin{lstlisting}[language=Python,caption=Verschlüsselungsprozess]
def encrypt_phonebook_data(phonebook_json, client_public_key):
    # 1. 48-Byte Secret generieren (16B IV + 32B AES Key)
    secret = generate_secret()
    
    # 2. Secret mit RSA verschlüsseln
    encrypted_secret = rsa_encrypt(secret, client_public_key)
    
    # 3. Daten mit AES verschlüsseln
    encrypted_data = aes_encrypt(phonebook_json, secret)
    
    return encrypted_secret, encrypted_data
\end{lstlisting}

\section{Kommunikationsabläufe}

\subsection{Client-Registration}

\begin{enumerate}
\item Client sendet REGISTER mit Public Key und Verify-Code
\item Server validiert Verify-Code und registriert Client
\item Server sendet Merkle Root aller Public Keys
\item Client validiert Merkle Tree Integrität
\end{enumerate}

\subsection{Anrufinitiierung}

\begin{enumerate}
\item Client A fordert Public Key von Client B
\item Server liefert Public Key mit Verify-Code
\item Client A verschlüsselt Session-Key mit Client B's Public Key
\item Server leitet verschlüsselte Daten an Client B weiter
\item Bei Annahme: Bidirektionale Audio-Streams werden etabliert
\end{enumerate}

\section{UDP-Audio-Relay}

\subsection{Architektur}

\begin{lstlisting}[language=Python,caption=UDP-Relay-Setup]
class CONVEY:
    def __init__(self, server_instance):
        self.udp_relay_port = 51822
        self.audio_relays = {}
        self._start_udp_relay()
    
    def _register_audio_relay(self, call_id, caller_ip, callee_ip):
        self.audio_relays[call_id] = {
            'caller_addr': (caller_ip, 51821),
            'callee_addr': (callee_ip, 51821)
        }
\end{lstlisting}

\subsection{Datenfluss}

\begin{enumerate}
\item Audio-Daten werden lokal mit AES-256-CBC verschlüsselt
\item Verschlüsselte Pakete werden an UDP-Relay gesendet
\item Relay leitet Pakete an Zielclient weiter
\item Empfänger entschlüsselt lokal mit Session-Key
\end{enumerate}

\section{Server-Discovery und Load-Balancing}

\subsection{Seed-Server Architektur}

\begin{lstlisting}[language=Python,caption=Relay-Manager]
class AccurateRelayManager:
    def __init__(self, server_instance):
        self.SEED_SERVERS = [
            ("sichereleitung.duckdns.org", 5060),
            ("sichereleitung.duckdns.org", 5061)
        ]
        self.known_servers = {}
        self.server_load = 0
\end{lstlisting}

\subsection{Client-seitige Server-Auswahl}

\begin{lstlisting}[language=Python,caption=Server-Discovery]
class ClientRelayManager:
    def discover_servers(self):
        for seed_host, seed_port in self.SEED_SERVERS:
            # Ping Server auf beiden Ports (5060/5061)
            # Wähle Server mit niedrigster Last und Ping-Zeit
            best_server = self._select_best_server()
\end{lstlisting}

\section{Erweiterte Audio-Features}

\subsection{Qualitätsprofile}

\begin{lstlisting}[language=Python,caption=Audio-Konfiguration]
QUALITY_PROFILES = {
    "highest": {"format": "S32_LE", "rate": 192000, "channels": 2},
    "high": {"format": "24-bit", "rate": 192000, "channels": 2},
    "middle": {"format": "24-bit", "rate": 48000, "channels": 2},
    "low": {"format": "16-bit", "rate": 48000, "channels": 1}
}
\end{lstlisting}

\subsection{Rauschfilterung}

\begin{enumerate}
\item 180-Sekunden Clear-Room Profilerstellung
\item Adaptive Rauschschwellen basierend auf RMS
\item Spektrale Rauschunterdrückung ohne scipy Abhängigkeit
\item Frequenzspezifische Filter für 50Hz Brummen
\end{enumerate}

\section{Sicherheitsanalyse}

\subsection{Bewusst offene Angriffsvektoren}

\begin{itemize}
\item \textbf{Verify-Code Seed}: Client-Name ist öffentlich → RNG-State-Angriffe möglich
\item \textbf{SIP-Header}: Traditionelle SIP-Verwundbarkeiten bleiben bestehen
\item \textbf{Timing-Angriffe}: Keine zusätzlichen Timing-Gegenmaßnahmen
\end{itemize}

\subsection{Erkennungsmechanismen}

\begin{itemize}
\item \textbf{RNG-State-Angriffe}: Verify-Code Wiederverwendung wird protokolliert
\item \textbf{Replay-Angriffe}: Counter-System verhindert Replay innerhalb Toleranz
\item \textbf{Man-in-the-Middle}: Merkle Tree validiert Schlüsselintegrität
\end{itemize}

\subsection{Protokollierte Angriffstypen}

\begin{lstlisting}[caption=Angriffsprotokoll-Beispiele]
# attacks.log Einträge:
🚨 [RNG_STATE_ATTACK] VERIFY_CODE_REUSE
❌ [INVALID] INVALID_VERIFY_CODE
⚠️ [MALFORMED] MALFORMED_VERIFY_CODE
⚠️ [MISSING] MISSING_VERIFY_CODE
📏 [FRAME] OVERSIZE (DDoS-Versuch)
\end{lstlisting}

\section{Performance-Optimierungen}

\subsection{Thread-Safety}

\begin{lstlisting}[language=Python,caption=Thread-sichere Datenstrukturen]
class VerifyGenerator:
    def __init__(self, seed, client_id):
        self._lock = threading.Lock()  # Pro Client-Instanz
    
    def verify_code(self, received_code):
        with self._lock:  # Thread-sicher
            # Verifikationslogik
\end{lstlisting}

\subsection{Queue-basierte Verarbeitung}

\begin{lstlisting}[language=Python,caption=Nachrichten-Queue]
def handle_server_message(self, raw_data):
    if not hasattr(self, '_message_queue'):
        self._message_queue = []
    
    self._message_queue.append({
        'type': 'frame_data',
        'data': raw_data,
        'timestamp': time.time()
    })
    
    # Asynchrone Verarbeitung
    self._process_queue_simple()
\end{lstlisting}

\section{Entwicklerhinweise}

\subsection{Experimentelle Features}

\begin{itemize}
\item RNG-Verifikation dient primär der Angriffserkennung, nicht -verhinderung
\item Bewusst einfache Implementierung für Analysezwecke
\item Extensive Protokollierung in \texttt{attacks.log} für Forensik
\end{itemize}

\subsection{Produktionseinsatz}

Für Produktionseinsatz empfohlen:
\begin{itemize}
\item Gemeinsames Geheimnis für Verify-Code Seed
\item Erhöhte Verify-Code Länge (8+ Zeichen)
\item Strengere Counter-Toleranz
\item Zusätzliche Timing-Gegenmaßnahmen
\end{itemize}

\begin{thebibliography}{9}
\bibitem{rfc3261} Rosenberg, J. \emph{SIP: Session Initiation Protocol}. RFC 3261.
\bibitem{merkle} Merkle, R. C. \emph{A Digital Signature Based on a Conventional Encryption Function}. CRYPTO 1987.
\bibitem{sha3} Dworkin, M. J. \emph{SHA-3 Standard: Permutation-Based Hash and Extendable-Output Functions}. FIPS PUB 202.
\end{thebibliography}

\end{document}